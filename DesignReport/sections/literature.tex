\section{Literature Review}
\subsection{Introduction}
In Belgium one man in three and one woman in four faces cancer before his or
her 75th birthday \cite{kanker}. In 2010 62 017 new cases of cancer were
diagnosed in Belgium \cite{kankerliga}. The World Health Organisation estimates
the worldwide death toll from lung cancer will be 10 000 000 by , which makes it
one of the deadliest cancers \cite{gu, zheng}.
However, an early detection can increase the survival rate up to 70-80\%
\cite{swensen}. Furthermore, research has shown that the detection of lung
cancer in an early stage broadens the amount of treatment options and increases
the amount of invasive surgery\cite{greenlee}.


Due to recent developments in
computed tomography (CT) technology it is now possible to obtain near isotropic,
submillimeter resolution images of the complete chest in a single breath hold.
This high resolution has the advantage that it enables visualisation of small
and low-contrast nodules that could hardly be screened in conventional
programs. The downside is that enormous amounts of data are generated
which increases the work load of radiologists, especially since low-dose CT
scans are more and more implemented in routine screenings. Still, this is no
idle measure. Long nodules are very commonly detected on CT scans. Research
shows that up to 51\% of smokers aged 50 years or older have pulmonary lung
nodules on CT scans \cite{mahon}. Therefore, the United States Preventive
Services Task Force for example stated that it ``recommends annual screening for
lung cancer with low-dose computed tomography (LDCT) in adults aged 55 to 80
years who have a 30 pack-year smoking history and currently smoke or have quit
within the past 15 years'' \cite{ups}.
Therefore, the detection of pulmonary nodules from volumetric computed
tomography (CT) scans if one of the most studied CAD applications
\cite{sluimer}.


Currently, expert radiologists perform the investigation of the
CT scans. They use the shape, the texture, the location and the growth rate of
the volume of the nodule as clinical parameter to determine the malignancy of
the nodules and to decide on the diagnosis of lung cancer. A jagged shape
nodule is more likely to be lung cancer than a smooth one. A fatty, bony,
watery nodule or a mixture of these different contents is less likely to
indicate lung cancer than a nodule that is attached to a vessel. A nodule
attached to the lung wall is typically diagnosed as benign if the
volume-doubling periode is longer than 400 days \cite{wu}. Nevertheless, the
examination of these scans is a time-consuming task and is not free from errors.
Although small nodules are in principle detectable in CT scans, a non-negligible
fraction may be overlooked if they are situated in a maze of vessels of similar size \cite{ozekes}.
FIGUUR
Another problem that arises is the intra- and interreader variability
amongst radiologists in pulmonary lung detection \cite{armato} \cite{hens}. Therefore,
there is a need for a computer-aided detection (CAD) system that can assist the
radiologist in the detection of pulmonary nodules.

\subsection{The biology of lung nodules}
FIGUUR vanginneken
Lung nodules are lung tissue abnormalities that are roughly spherical with a
diameter up to 30 mm. On chest CT scans they appear as a rounded or
irregular opacity. Many types of lung nodules can by distinguished on CT scans.
A centrilobular nodule is separated by several millimeters from the pleural
surfaces, fissures and interlobular septa. They range in size from a few
millimeters to 10 millimeters.
A micronodule is less than 3 millimeters in diameter. A ground-glass nodule --
or non-solid nodule -- appears on the CT scans as a hazy attenuation in the
lung. This type of nodule does not efface the bronchial and vascular margins. A solid
nodule shows a homogeneous soft-tissue attenuation. Finally, a part-solid nodule
exhibits both ground-glass and solid soft-tissue attenuation characteristics
\cite{nodule}. 

The types of nodules stated above can be categorised. Juxta-vascular
pulmonary nodules have significant connections to their neighbouring vessels.
Pleural tail nodules have only thin connections to the neighbouring pleural
wall. Well-circumscribed nodules on the other hand do not have a connection to
the neighbouring vessels and structures. Juxta-pleural nodules show some degree
of attachment to their neighbouring pleural surface \cite{kostis}.

A number of nodule segmentation algorithms perform well in detecting specific
types of nodules e.g. large, spherical, isolated nodules. However, these CAD
systems show large limitations in detecting e.g. non-isolated nodules that are
connected to the pulmonary wall \cite{keshani}. These algorithms can be usefull
in particular situations, but if a detection algorithms really aims at being an
asset for the radiologist, it should be able to detect all nodules while
refusing as much false positives as possible.

\subsection{Overview of existing lung nodule detection systems}
As the demand for a reliable CAD system to detect pulmonary nodules is urgent, a
lot of research has been dedicated to the matter. Several commercial systems
have already been developed and many workstations that radiologists use to
examinate CT scans offer on-board nodule detection or enhancement capabilities
\cite{ginneken}. However, although a lot of efforts were made, the results shown
in the various studies are rather diverse.

\subsubsection{Commercial systems}
In 2004 iCAD, Inc., provided lung cancer detection, analysis and tracking
software for the TeraRecon's Aquarius product line. The latter licensed three
software modules from iCAD. The iCAD QuickCueTM for example automatically
detects cancerous lung nodules while the iCAD QuickMatchTM locates, compares and tracks these
nodules in previous or subsequent patient studies \cite{tera}.
However, the ImageChecker CT, launched by R2 Technology, was the first CT Lung
CAD system approved by the US Food and Drug Administration for the detection of
lung nodules during the examination of CT scans \cite{Mevis}. In 2005 R2
Technology, Inc., introduced the second-generation ImageChecker CT Lung Version
2.0 CAD system which also implemented the AutoPoint temporal comparison
algorithm. This CAD system ``highlights abnormalities'' and compares new and
past images to demonstrate changes that have occured over time \cite{diag, r2}.
In 2006 Vital Images, Inc., and R2 Technology, Inc., announced the
implementation of the R2 Technology's ImageChecker CT Lung CAD software into
the Vitrea workstations \cite{vital}. In 2006 anohter company, Hologic, Inc.,
acquired R2 Technology, Inc., and implemented their CAD technology
\cite{Hologic}. Then, in 2008, MeVis Medical Solutions AG, Inc., acquired the
Pulmonary Computed Tomography Business from Hologic R2, Inc. \cite{Mevis}. 


Although the technology from R2 Technology is highly demanded, some companies
have also developed their own software. In 2007 Medicsight plc announced it was
granted a medical device license from the Therapeutic Products Directorate of
Health in Canada to introduce Medicsight LungCAD API \cite{HI}.
Median Technologies offers the LMS-Lung and LMS-Lung/CAD modules which provide
quantification and detection functionalities for pulmonary (solid) nodules and
micronodules \cite{median}. And Siemens has developed the syngo.CT Lung CAD.
They claim it is ``a fully automated computer assisted second reader tool'' that
is designed to assist radiologists in the detection of solid pulmonary nodules \cite{siemens}.

\subsubsection{Automated lung detection systems: publications}
Apart from these numerous commercial systems also a lot of academic research
centra have tried to come up with a successful pulmonary nodule detection
system. \cite{review} suggest that most CAD systems for the automated detection
of lung nodules proceed according to a number of steps of which the first one is
the acquisition of data. The detection of lung nodules is preferably performed
in CT scans as they enable the visualisation of small volume and low-contrast
nodules because of the limited slice thichness. A large number of chest CT scans
are available in public databases such as the LIDC database. During the next
step, the data are pre-processed to remove noise and artefacts. This might
improve the quality of the images, but it is not necessary to do so. In the
third step a segmentation of the lungs is performed. The lung lobe region is
identified and the rest of the image is removed. This reduces the computational
cost compared to the case where the whole image is processed and it increases
the reliability, the accuracy and the precision of the algorithm. This
increases the performance of the next step: the nodule detection. ``Lung nodule
detection refers to the process of determining whether nodule patterns are
present in the image and identifying the location of the nodules''
\cite[p.~154]{review}. Nodule detection methods can be categorized according to
the detection method that is applied. The first group of publications uses a
classification technique to classify voxels or regions of interest (ROI). In
addition, a clustering method may be implemented to improve the perfromance of
the automated nodule detection method. The second group uses template matching
to detect specific geometries. The third group relies entirely on the output of
a lung nodule segmentation method. The systems that include a classification
component in their nodule detection algorithm have demonstrated better
performances \cite{review}. In the final step, the amount of false positives is
reduced to achieve a maximum sensitivity. In the following paragraphs a summary
of relevant literature is given.

The first problem that arises when laymen start processing CT scan in search
for nodules is that they have to rely on the annotations made by expert
radiologists. Accurate delineation of these lung abnormalities is crucial for
optimal image analysis. The current approach to delineate lung nodules in CT
scans involves one or more radiologists manually drawing the boundaries of the
nodules. Often this manual segmentation overestimates the nodule volume to
ensure the entire lesion is enclosed \cite{rex}. Furthermore, this process is
highly variable \cite{cooper}. But the success of the extraction of image
features depends on the accurate delineation of the nodules. Therefore, if one
wants to develop a nodule segmentation method -- or any other nodule detection
method -- is it of major importance this delineation is done in an accurate and
reproducible way. \cite{gu} have improved the ``Click and Grow'' algorithm that
has been developed by Definiens AG and Merck and Co., Inc., which
semi-automatically isolates tumors in CT images. The idea is that a radiologist
detects the nodule and clicks on the region of interest in a 2D slice. This
click initiates multiple seed points in a certain area. Then the application
builds out the object three-dimensionally by region growing. An ensemble
segmentation is obtained from the multiple regions that were grown. An
evaluation on a set of 129 CT scans using a similarity index (SI) was performed.
The average SI was above 93\% which shows stability of the algorithm. The
average SI for two different readers was 79,5\%.

\cite{elbaz} proposed an three step algorithm to isolate lung nodules from
spiral chest low-dose CT (LDCT) scans. In the first step the lung tissues were
isolated by applying an iterative Markov-Gibbs-random-field (MGRF) based
segmentation framework. To retain the nodules attached to the pulmonary walls, the
segmentation was refined by the iterative conditional mode relaxation that
maximizes a MGRF energy. Then the lung nodules, arteries, veins, bronchi and
bronchioles were separated from the rest of the tissues in the slice. In the
second step the lung nodules (2-12 mm) were detected by applying 3D and 2D
templates which describe typical geometry and greylevel distributions within
nodules of the same type. Four template shapes were used: solid sphere, hollow
sphere, solid circle and solid semicircle. The radius and the greyscale
intensity of the templates was made adaptable. The detection combined the
normalized cross-correlation template matching and a genetic optimization
algorithm. The third step eliminated the false positive nodules using three
textural and geometric features that were calculated for each detected nodule.
To distuingish between false positives and true positives Bayesian supervised
classifier learning statistical characteristics from a training set (20 FP, 20
lung TP, 20 lung wall TP nodules) of nodules selected from 50 separate subjects.
CT scans from 200 subjects were used in this study. The sensitivity was 82,3\%
and a FPNs rate of 9,2\% (i.e. the number of FPNs -- 12-- with respect to the
total number of true nodules -- 130 -- ). The speed of the execution is a
function of the CPU and the data size. The algorithm, that was implemented in
C++ on an Intel dual processor wiht 16 GB memory and 2 TB hard drive, took about
5 minutes to process 182 CT slices of size 512 x 512.

Other studies rely on a nodule segmentation method to detect lung abnormalities.
\cite{kuh} applies morphological opening, erosion, thresholding, seed
optimisation and boundary refinement operations to extract large nodules.
\cite{itai} proposes a segmentation of the lung areas using SNAKES method which
is an active contour model. Abnormal shadow areas including ground-glass opacity
or lung cancer over the size of 5 mm are classified by using voxel densities.
The algorithm was applied on 9 CT scans and a true positive fraction of 0,81 and
a false negative fraction of 0,2 were obtained.

A third category of nodule detection methods are the classification based
methods. \cite{ozekes} tested four different learning based classification
methods: a Neural Network classifier, a Support Vector Machine (SVM) classifier,
a naive Bayes classifier and a logistic regression classifier. First ROIs were
extracted by applying thresholding and an 8-directional search in which
candidate lung nodule voxels have to have neighbour voxels with densities
between a minimum and maximum density threshold. From these ROIs a number of
features were extracted: straightness, thickness, vertical and horizontal
widths, regularity and vertical and horizontal black pixel ratios. These
features were then fed to the four classifiers. The Neural Network classifier
showed the best results, followed by the SVM classifier. \cite{keshani} also
applies an SVM classifier and active contour modelling to detect lung nodules.
The lung area was first segmented by active contour modelling which was followed
by a set of masking techniques to transfer nodules from non-isolated into
isolated ones. Based on a set of 2D and 3D features the SVM classifier detected
the lung nodules. Then the contours of these nodules were extracted
by active contour modelling. In a last step the lung tissues in the original
image were classified into four classes: lung wall, parenchyma, bronchioles and
nodules. The results form this classification were used to distuingish solitary
nodules from attached ones. When this algorithm was used to detect nodules in
the ANODE09 dataset an average detection rate of 37,8\% was obtained while the
best performing method yielded a detection rate of 63,2\%. The latter algorithm
was developed by \cite{mur} and uses the local image features -- shape index and
curvedness -- to detect candidate nodules. Two successive k-nearest-neighbour
classifiers are applied to reduce the number of false positives. This yielded a
sensitivity of 80\% with an average of 4,2 false positives per scan.

lee (2010) PAPER

\subsubsection{Performance of existing systems}
 The algorithms presented in a wide range of papers report varying
 successes in the automated detection of nodules. However, it is very difficult
 to compare studies against one another in a meaningful way due to differences
 in the size of the datasets, the evaluation methods, the data selection
 criteria and the characteristics of the nodules \cite{lee2010}. Especially
 comparing older and contemporary studies is difficult as older ones may have
 used scans with thicker sections (range 2.5 - 10 mm), on which small nodules
 are rather difficult to detect, than the scans nowadays (2,5 mm) \cite{lee2010,
 ginneken} \cite{mur}. Some studies focus on nodules below or above a certain
 size or on special types of nodules (e.g. solid nodules). \cite{mur} performed an extended
 literature review and found that the number of scans used for testing varied between 5 and 500 with a
 median number of 29,5. Many of the studies included multiple scans from
 individual patients, which means that the diversity of the available nodules
 was reduced. Furthermore, the results of publications are often presented in a
 diverse way. \cite{results} suggests the performance of CAD systems should be
 presented a the sensitivity of the system, the specificity, the accuracy, the
 error rate, the True Positive Rate and the False Positive Rate.
 
 TABEL X GEEFT OVERZICHT WEER VAN BEST AND SLECHTS PRESTERENDE ALGORITHMS BASED
 ON EXHAUSTIVE LITERATURE REVIEW
 
 In order to improve the access to data, and hereby the comparability between
 studies, the Lung Image Data Consortium (LIDC) created a publically available
 database which provides researchers with a vast amount of test- and
 trainingsdata. Nevertheless, as one can take different subdatasets from this
 large database for the training and the testing of algorithms, it is still not
 possible to compare results in an objective and meaningful way. Therefore,
 \cite{ginneken} created ANODE09, a database of 55 scans and a web-based
 framework which allows researchers to test their algorithms and to
 compare results against one another.
 
\subsection{Ensemble classifiers}
 methods met classifiers presteren beter (lee, 2010)
- reden om classifiers te gaan gebruiken

\subsubsection{Introduction to ensemble classifiers}
voordelen/nadelen verschillende classifiers

waarom random forest

\subsubsection{Random Forests}
wisk achtergrond







