\section{Algorithms}

\subsection{Datasets}
The RF algorithm was trained and tested on \ldots and \ldots CT scans,
subsampled from the LIDC/IDRI database (freely available at
http://cancerimagingarchive.net/), consisting of \ldots and \ldots slices
respectively. The pixel size of the scans varied between 0,586 to 0,963 mm,
while the slice thickness varied between 1,25 or 2,5 mm. Together with the
original DICOM images the associated XML files were obtained. These XML files
provided a set of characteristics for each nodule found: region, subtlety,
spiculation, internal structure, spiculation, lobulation, shape (sphericity),
solidity, margin, and likelihood of malignancy \cite{lidcbase}.

The LIDC/IDRI database consists of 1018 thoracic CT scans that are obtained from
a heterogeneous range of scanner models (seven GE Medical Systems LightSpeed
scanner models, four Philips Brilleance scanner models, five Siemens Definition,
Emotion, and Sensation scanner models and one Toshiba Aquilion scanner model).
The database includes only one scan per patient so the scans are not correlated.
The nodules in the scans were delineated by at least four different expert
radiologists to indentificate as much nodules as possible. For this purpose the
indentification process was also subdivided into two phases: a blinded read
phase and an unblinded read phase. During the initial blinded read phase each
radiologist independently reviewed all scans and indicated the nodules in the
range of 3 to 30 mm, the nodules smaller than 3 mm (if not clearly benign) and
the non-nodules (other pulmonary lesions e.g. an apical scar) larger or equal to
3 mm. In the subsequent unblinded read phase the anonymized blinded read results
of all radiologists were revealed to each of the radiologists who then
independently reviewed their marks along with the anonymous marks of their
colleagues. The delineation of the nodules was done completely manually or in a
semiautomated way. This was allowed as a study on this topic showed that the
variation in nodule delineation done by different radiologists
substantially exceeded the variation derived from different software tools
\cite{lidcbase}.






welke materialen?
\begin{enumerate}
\item welke datasets gebruikt: lidc + annotaties van experts
\item aantal (scans + slices)
\item eigenschappen (512x512, pixelsize, slice thickness)
\item nodules in scans (aantal, grootte)
\item welke computers gebruikt
\item welke programma's gebruikt (voor wat? mevislab: exploratie + region
growing; python)
\end{enumerate}

\subsection{Training phase}

klassendiagram: hoe is programma opgebouwd?
TRAINING + TEST
\begin{enumerate}
\item stappen die doorlopen worden + fig
\item welke algorithmes gebruikt + wat doen ze + waarom die keuze?
\item vergelijken keuzes met literatuur/commerciele systemen?
\item crossvalidatie
\end{enumerate}

\subsection{Validation phase}

trainingstage + teststage
tussenresultaten: moeilijkheden, opl, \ldots


VERGELIJKEN MET SVM?